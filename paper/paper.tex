\documentclass{article} % For LaTeX2e
\usepackage{nips14submit_e,times}
\usepackage{hyperref}
\usepackage{url}
\usepackage{amsmath,amsfonts,amsthm}
\usepackage{bbm}
\usepackage{algorithm,algorithmic}
\usepackage{graphicx}
\usepackage{bm}
\usepackage{bbm}
\usepackage[titletoc]{appendix}
\usepackage{wrapfig}
\usepackage{afterpage}
\usepackage{amssymb}
\usepackage{booktabs}
\usepackage{ulem}
\usepackage{multirow}

\def\B#1{\bm{#1}}
%\def\B#1{\mathbf{#1}}
\def\trans{\mathsf{T}}

%\renewcommand{\labelitemi}{--}

\newtheorem{theorem}{Theorem} \newtheorem{lemma}[theorem]{Lemma}
\newtheorem{proposition}[theorem]{Proposition}
\newtheorem{corollary}[theorem]{Corollary}
\newtheorem{definition}[theorem]{Definition}
\newtheorem{remark}{Remark}

%%%%%%%%%%%%%%%%%%%%%%%%%%%%%%%%%%%%%%%%%%%%%%%%%%%%%%%%%%%%%%%%%%%%%%%%%%%%%%%

\title{Introducing region-network priors to statistical estimators for fMRI data}

\newcommand{\fix}{\marginpar{FIX}}
\newcommand{\new}{\marginpar{NEW}}
\DeclareMathOperator{\proj}{proj}
\DeclareMathOperator{\softmax}{softmax}
\DeclareMathOperator{\prox}{prox}
\DeclareMathOperator{\Prox}{Prox}
\DeclareMathOperator{\im}{im}

% macros from michael's .tex
\DeclareMathOperator{\dist}{dist} % The distance.
\DeclareMathOperator{\argmin}{argmin}
\DeclareMathOperator{\argmax}{argmax}
\DeclareMathOperator{\Id}{Id}
\DeclareMathOperator{\abs}{abs}
\newcommand{\R}{\mathbb{R}}
\newcommand{\N}{\mathbb{N}}
\newtheorem{thm}{Theorem}[section]
\newtheorem{prop}[thm]{Proposition}
\newtheorem{lem}[thm]{Lemma}
\newtheorem{cor}[thm]{Corollary}


\newcommand{\suggestadd}[1]{{\color{blue} #1}}
\newcommand{\suggestremove}[1]{{\color{red} \sout{#1}}}

% \nipsfinalcopy % Uncomment for camera-ready version
\nipsfinaltrue
%%%%%%%%%%%%%%%%%%%%%%%%%%%%%%%%%%%%%%%%%%%%%%%%%%%%%%%%%%%%%%%%%%%%%%%%%%%%%%%

\begin{document}

\author{Danilo Bzdok, Michael Eickenberg, Olivier Grisel,
  Ga\"el Varoquaux, Bertrand Thirion
  INRIA, Parietal team, Saclay, France\\
  CEA, Neurospin, Gif-sur-Yvette, France\\
  firstname.lastname@inria.fr}

\maketitle

\begin{abstract}
Imaging neuroscience links human behavior to aspects of brain
biology in ever-increasing datasets.


\textbf{\\keywords}:
Brain Imaging, Functional Specialization, Functional Integration,
Sparsity-inducing Norms, Numerical Optimization, Systems Neuroscience

\end{abstract}



\section{Introduction}
% sparsity
Many quantitative scientific domains underwent a
recent passage from the classical regime (i.e., "long data")  to
the high-dimensional regime (i.e., "wide data")
\citep{jordan2015massive}.
Also in the brain imaging domain,
many contemporary methods for acquiring brain signals yield
more variables per observation than
total observations available in a data sample.
This scenario challenges many statistical estimators from
classical statistics.
For instance,
linear regression models without additional assumptions
yield an infinity of possible coefficients
and, thus, no solution.
%
\textit{Sparsity} assumptions have consequently been
introduced and made many ill-posed estimation problems tractable
\cite{buhlmann2011statistics, hastie2015statistical}.
Domain-specific structural knowledge can be imposed on the 
statistical estimation by preassuming that not all variables will be
equally important
\citep{bach2012optimization}.
Sparsified supervised and unsupervised
learning algorithms have proven to yield
statistical relationships that can be readily
estimated, reproduced, and interpreted
\cite{giraud2014introduction}.
%
Yet, what is the most pertinent neurobiological structure
that can successfully tackle the
\textit{curse of dimensionality} in
functional neuroimaging research?



% specialization & integration
Concepts on human brain organization have long been torn
between the two extremes
\textit{functional specialization} and \textit{functional integration}.
Functional specialization emphasizes that microscopically distinguishable
brain regions probably solve distinct classes of computational processes
\citep{kanwisher2010functional}.
Functional integration, in turn, emphasizes that brain function
is probably enabled by complex connections between these
distinct brain regions.
%
This notion was predominantly derived from
invasive examination of \textit{structure} (i.e., histological preparation),
\textit{connectivity}, (i.e., axonal tracing),
and \textit{functional properties}
(i.e., single cell recordings) in same animals.
Regarding functional segregation into specialized regions,
early histological investigations into the microscopic heterogeneity of
the human cerebral cortex have resulted
in several detailed anatomical maps
\citep{{brodmann1909vergleichende, vogt1919allgemeine}.
Regarding axonal connections,
each such cortical area has been observed
to possess a unique set of incoming and outgoing connections
\citep{passingham2002, young93monkey, scannell95cat}.
%
In particular,
the realization of computational processes (i.e., function)
is probably a consequence of both
local cyto- and chemoarchitectonic infrastructure
and its unique global connectivity profile.
While
cortical modules versus connections between those
reflect 
functional specialization versus functional integration,
\citep{friston2002beyond},
these architectural principles are conceptually inextricable in
the realization of mental operations
\citep{tononi1998complexity}.



method -> type of discovery


SPECIAL
Functional specialization has been investigated in many ways.
%
Single cell recordings and microscopic examination
revealed, for instance, the
specialization in the visual cortex into V1, V2, V3, V3A, and V4
\citep{hubel1962receptive, zeki1978functional}.


example: fusiform gyrus
-> lesion -> does not work anymore


Non-invasive brain imaging with
functional magnetic resonance imaging (fMRI) and
positron emission tomography (PET)
enabled the localization of
sensory, motor, and emotional functions to cortical areas
in the living brain
\citep{fristen1997imaging}.



Radioactive mapping the distribution of
neurotransmitter receptors added yet another
local characteristic of neuronal populations
\citep{zilles2009receptor}.


clustering
\citep{behrens03}


High-thoughput computing enabled an 
ultahigh-resolution 3D model of brain anatomy
at macroscopical to near-cellular scale
\citep{amunts2013bigbrain}.



=> methods: non-overlapping, discrete region units




INTEGRAL


Only recently we have shifted the main interpretational focus
from single region to macroscopical networks
in systems neuroscience \citep{yuste2015}.


eemrgent properties emerge from the interaction

In
contrast, functional integration emphasizes that function is an
emergent property of collaborating areas. In fact, the higher the
hierarchical level, the more one cortical area connects to different
large-scale networks, which makes their boundaries less clear
(Yeo/Buckner 2011). The above network-network models naturally embody
functional integration as they capture (partially overlapping)
patterns of spatial coherence.

Systems neuroscience has established existence of a
set of fluctuating yet robust brain networks in humans. 

example: DMN network?


Identical neural networks have reappeared across cognitive domains
using diverging methods. These observations prompted widely-adopted
notions, including the ``default-mode network'' \cite{raichle2001} (DMN; Raichle et al.,
2001), ``salience network'' (SN; Seeley et al., 2007), and ``dorsal
attention network'' (DAN; Corbetta and Shulmann,
2002). Developmentally, such large-scale networks emerge during late
fetal growth (Doria et al., 2010), before cognitive capacities mature
in childhood. In adults, nodes of a same cohesive network have more
similar functional profiles than nodes from different networks
(Anderson et al., 2013).
This
suggests network-network relationships as an under-appreciated unit of
functional brain organization.



Onset of a given cognitive task might induce
characteristic changes in functional coupling between large-scale
networks. Indeed, a working-memory task entailed increase in metabolic
activity in DAN regions but decrease in default-mode regions (Fransson
et al., 2006). Notably, the functional connectivity did not change
significantly within either DAN or DMN. Additionally, during auditory
event transitions in another experimental fMRI study, both DAN and SN
increased in activity, whereas the DMN decreased in activity
(Sridharan et al., 2008). 
 

Such mediation between supraordinate networks
has been proposed to involve the right anterior insula (Sridharan et
al., 2008) and right temporo-parietal junction (Bzdok et al.,
2013).
The relevance of networks
possibly extends to psychiatric, neurological, and neurodegenerative
disorders (Seeley et al., 2009; Menon 2011).

=> methods: overlapping, continuous network units
invasive axonal tracing studies,
ICA,
seed-based analyses of BOLD fluctations or
electrophysiological oscillations,
as well as
effective connectivity models
and
graph-theoretical measurements
\citep{friston03dcm, buzsaki2004neuronal,
beckmann2005, bullmore2009complex, jbabdi2013long}.


analysis and integration of information





% study
The present study attempts a combination of specialization and
integration aspects by means of structured sparsity
for high-dimensional inference.
%

adaptive estimator
biologically plausible

recent advances in high-dimensional statistics enable

The present investigation reframed network-network dynamics as a
multivariate statistical learning problem.
capitalized on two independent, large datasets (n=500 and n=80) with
18 typical neuroimaging tasks each, covering a spectrum of humans'
cognitive apparatus. 

determined a plausible, quantitative
combination of the brain signal's modes of variation. This allowed
avoiding the curse of dimensionality and making the machine-learning
results human-interpretable by quantitative association with
psychological concepts.

neurobiologically motivated restrictions to complexity circumvented
the curse of dimensionality and
allowed for useful, simplified views on brain function.

importance of brain regions and networks is probably a matter of
degree.
We estimate that degree from the data

help reintegrate the divorced
interpretational streams based on
region- and network-focused neuroscientific investigations.




\section{Methods}
%
\paragraph{Rationale}

we need to inject domain knowledge into
statistical estimations to harness the curse of dimensionality.


should be able to estimate voxel level
while taking into account known supravoxel structure.



%
\paragraph{Data.}
As the currently biggest openly-accessible reference dataset,
we chose resources from the Human Connectome Project (HCP)
\cite{barch2013}.
Neuroimaging task data with labels of ongoing cognitive processes
were drawn from 500
healthy HCP participants (cf. Appendix for details on datasets).
18 HCP tasks 
were selected that are known to elicit reliable neural activity
across participants (Table \ref{table_tasks}).
In sum, the HCP task data incorporated 8650 first-level activity maps
from 18 diverse paradigms administered to 498 participants (2 removed
due to incomplete data).
All maps were resampled to a common $60\times72\times60$ space of
3mm isotropic voxels and gray-matter masked (at least 10\% tissue
probability).
The supervised analyses were thus based on labeled HCP task maps with
79,941 voxels of interest representing z-values in gray matter.

\begin{table}[h]
  \resizebox{0.98\textwidth}{!}{%
  \begin{tabular}{l|l|l}
    \hline
  {\bf Cognitive Task} & {\bf Stimuli}                         & {\bf Instruction for participants}                                                \\ \hline
  1 Reward             & \multirow{2}{*}{Card game}            & \multirow{2}{*}{Guess the number of a mystery card for gain/loss of money}        \\ \cline{1-1}
  2 Punish             &                                       &                                                                                   \\ \hline
  3 Shapes             & Shape pictures                        & Decide which of two shapes matches another shape geometrically                    \\ \hline
  4 Faces              & Face pictures                         & Decide which of two faces matches another face emotionally                        \\ \hline
  5 Random             & \multirow{2}{*}{Videos with objects}  & \multirow{2}{*}{Decide whether the objects act randomly or intentionally} \\ \cline{1-1}
  6 Theory of mind     &                                       &                                                                                   \\ \hline
  7 Mathematics        & Spoken numbers                        & Complete addition and subtraction problems                                        \\ \hline
  8 Language           & Auditory stories                      & Choose answer about the topic of the story                                        \\ \hline
  9 Tongue movement    & \multirow{3}{*}{Visual cues}          & Move tongue                                                                       \\ \cline{1-1} \cline{3-3} 
  10 Food movement     &                                       & Squeezing of the left or right toe                                                \\ \cline{1-1} \cline{3-3} 
  11 Hand movement     &                                       & Tapping of the left or right finger                                               \\ \hline
  12 Matching          & \multirow{2}{*}{Shapes with textures} & Decide whether two objects match in shape or texture                             \\ \cline{1-1} \cline{3-3} 
  13 Relations         &                                       & Decide whether object pairs differ both along either shape or texture             \\ \hline
  14 View Bodies       & Pictures                              & Passive watching                                                                   \\ \hline
  15 View Faces        & Pictures                              & Passive watching                                                                   \\ \hline
  16 View Places       & Pictures                              & Passive watching                                                                   \\ \hline
  17 View Tools        & Pictures                              & Passive watching                                                                   \\ \hline
  18 Two-Back          & Various pictures                      & Indicate whether current stimulus is the same as two items earlier                \\ \hline
  \end{tabular}
}
\vspace{-0.2cm}
\caption{\textbf{Description of psychological tasks to predict.}}
\label{table_tasks}
\end{table}

These labeled data were complemented by unlabeled activity maps
from HCP acquisitions of unconstrained resting-state activity
\cite{smith2013resting}.
These reflect brain activity in the absence of controlled thought.
In sum, the HCP rest data concatenated
8000 unlabeled, noise-cleaned rest maps with
40 brain maps from each of 200 randomly selected participants.

We were further interested in the utility of the
optimized low-rank projection
in one task dataset for dimensionality reduction in another task dataset.
To this end, the HCP-derived network decompositions were used as preliminary
step in the classification problem of another large sample.
The ARCHI dataset \cite{pinel07} provides activity maps from
diverse experimental tasks, including auditory and visual perception, motor action,
reading, language comprehension and mental calculation.
Analogous to HCP data, the second task dataset thus incorporated 1404
labeled, grey-matter masked, and z-scored activity maps
from 18 diverse tasks acquired in 78 participants.





sparse statistical models have only few nonzero parameters












\paragraph{Implementation.}
The analyses were performed in Python.
We used \textit{nilearn} to handle
the large quantities of neuroimaging data 
\cite{abrah14}
and
\textit{Theano} for automatic, numerically stable
differentiation of symbolic computation graphs
\cite{bastien2012theano, bergstra2010theano}.
All Python scripts that generated the results are
accessible online for reproducibility and reuse
% (\url{http://github.com/anonymous/anonymous}).
(\url{http://github.com/banilo/nips2015}).



\section{Experimental Results}
\paragraph{Serial versus parallel structure discovery and classification.}



\section{Discussion}

Functional specialization and functional integration
in the human brain are
probably both mechanistically relevant for emerging behavior.
However, there is a scarcity of statistical methods
that concomitantly relate brain regions and networks to
an organism's repertoire of mental operations.



neurobiologically motivated restrictions to
complexity circumvented the curse of dimensionality and allowed
for useful, simplified views on brain function.






\paragraph{Acknowledgment}
{\small The research leading to these results has received funding from the
European Union Seventh Framework Programme (FP7/2007-2013)
under grant agreement no. 604102 (Human Brain Project).
Data were provided by the Human Connectome Project.
Further support was received from
the German National Academic Foundation (D.B.)
and the MetaMRI associated team (B.T., G.V.).
}

\small
\bibliographystyle{splncs03}
\bibliography{paper_refs}



\end{document}

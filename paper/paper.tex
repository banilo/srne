\documentclass{article} % For LaTeX2e
\usepackage{nips14submit_e,times}
\usepackage{hyperref}
\usepackage{url}
\usepackage{amsmath,amsfonts,amsthm}
\usepackage{bbm}
\usepackage{algorithm,algorithmic}
\usepackage{graphicx}
\usepackage{bm}
\usepackage{bbm}
\usepackage[titletoc]{appendix}
\usepackage{wrapfig}
\usepackage{afterpage}
\usepackage{amssymb}
\usepackage{booktabs}
\usepackage{ulem}
\usepackage{multirow}

\def\B#1{\bm{#1}}
%\def\B#1{\mathbf{#1}}
\def\trans{\mathsf{T}}

%\renewcommand{\labelitemi}{--}

\newtheorem{theorem}{Theorem} \newtheorem{lemma}[theorem]{Lemma}
\newtheorem{proposition}[theorem]{Proposition}
\newtheorem{corollary}[theorem]{Corollary}
\newtheorem{definition}[theorem]{Definition}
\newtheorem{remark}{Remark}

%%%%%%%%%%%%%%%%%%%%%%%%%%%%%%%%%%%%%%%%%%%%%%%%%%%%%%%%%%%%%%%%%%%%%%%%%%%%%%%

\title{Introducing region-network priors
to statistical estimators for fMRI data}

\newcommand{\fix}{\marginpar{FIX}}
\newcommand{\new}{\marginpar{NEW}}
\DeclareMathOperator{\proj}{proj}
\DeclareMathOperator{\softmax}{softmax}
\DeclareMathOperator{\prox}{prox}
\DeclareMathOperator{\Prox}{Prox}
\DeclareMathOperator{\im}{im}

% macros from michael's .tex
\DeclareMathOperator{\dist}{dist} % The distance.
\DeclareMathOperator{\argmin}{argmin}
\DeclareMathOperator{\argmax}{argmax}
\DeclareMathOperator{\Id}{Id}
\DeclareMathOperator{\abs}{abs}
\newcommand{\R}{\mathbb{R}}
\newcommand{\N}{\mathbb{N}}
\newtheorem{thm}{Theorem}[section]
\newtheorem{prop}[thm]{Proposition}
\newtheorem{lem}[thm]{Lemma}
\newtheorem{cor}[thm]{Corollary}


\newcommand{\suggestadd}[1]{{\color{blue} #1}}
\newcommand{\suggestremove}[1]{{\color{red} \sout{#1}}}

% \nipsfinalcopy % Uncomment for camera-ready version
\nipsfinaltrue
%%%%%%%%%%%%%%%%%%%%%%%%%%%%%%%%%%%%%%%%%%%%%%%%%%%%%%%%%%%%%%%%%%%%%%%%%%%%%%%

\begin{document}

\author{Danilo Bzdok, Michael Eickenberg, Olivier Grisel,
  Bertrand Thirion,
  Ga\"el Varoquaux\\
  INRIA, Parietal team, Saclay, France\\
  CEA, Neurospin, Gif-sur-Yvette, France\\
  firstname.lastname@inria.fr}

\maketitle

\begin{abstract}
Imaging neuroscience links human behavior to aspects of brain
biology in ever-increasing datasets.


% OUR keywords
%\textbf{\\keywords}: curse of dimensionality; semi-supervised learning;
%fMRI; systems neuroscience

% official NIPS keywords
\textbf{\\keywords}: Brain Imaging, Functional Specialization,
Functional Integration,
Sparsity-inducing Norms, Systems Biology

\end{abstract}

\section{Introduction}
%
Many quantitative scientific domains underwent a
recent passage from the classical regime (i.e., "long data")  to
the high-dimensional regime (i.e., "wide data")
\citep{jordan2015massive}.
Also in the brain imaging domain,
many contemporary methods for acquiring brain signals yield
more variables per observation than
total observations available in a data sample.
This scenario challenges many statistical estimators from
classical statistics.
For instance,
linear regression models without any additional assumptions
yield an infinity of possible coefficients
and, thus, no solutions.
%
\textit{Sparsity} assumptions have consequently been
introduced and made many ill-posed estimation problems tractable
\cite{hastie2015statistical}.
Domain-specific structural knowledge can be imposed on the 
statistical estimation by preassuming that not all variables will be
equally important.
Sparsified statistical estimators have proven to yield
statistical relationships that can be
estimated, reproduced, and interpreted
\cite{giraud2014introduction}.
%
Yet, what would be the most pertinent neurobiological structure
that can successfully tackle the curse of dimensionality in
functional neuroimaging research?



abc
in current fMRI,
large-scale integration prevails over regional specialization,
although both notions are inextricably linked (Tononi et al., 1998).

It however remains elusive how brain regions and networks relate to
the organism's repertoire of mental operations.

Both regions and netoworks are
probably mechanistically relevant for emerging behavior.


The organization of the cerebral cortex into distinct modules may be described along several dimensions, most importantly, structure, connectivity and function. This conclusion was mainly derived from concurrent invasive examination of microstructure (histological preparation), connectivity (axonal tracing) and functional properties (single cell recordings) in same animals. 
Regarding the (micro-)structural dimension, early histological investigations into the microscopic heterogeneity of the human cerebral cortex have resulted in several detailed, though partially incongruent, anatomical maps (Brodmann 1909; Vogt and Vogt 1919). More recent advances in structural brain mapping have led to the development of observer-independent probabilistic cytoarchitectonic maps in stereotaxic space (Zilles and Amunts 2010). Moreover, the constantly growing field strength of fMRI scanners permits increasingly fin-grained microstructural recordings of the living human brain (Walters et al. 2007).
Regarding the connectional dimension, each cortical area is assumed to possess a unique set of input and output connections. Axonal connectivity between areas can be revealed by injection of a tracing dye that is transported to interconnected brain regions in animals. Axonal tracing studies in primates suggest that cortical areas tend to be connected hierarchically and reciprocally by feed-forward and feed-backward connections (Maunsell and van Essen 1983). Mapping connectivity patterns of a particular part of the cortex may thus allow identifying cortical areas by demonstrating differences between the connections of neighbouring grey matter locations.


Slightly reframing the concept of structure, connectivity and function as three pillars of brain organization, regional specialization of computational processes (i.e., function) can be conceived as being a consequence of both the local (micro-)structural and global connectional properties, as suggested by research in animals. That is, specialization of a particular function is not regarded as an intrinsic property of a brain region that is independent of its connectivity. Rather, input and output connectivity of an area in combination with the local "infrastructure" provided, e.g., by cyto- and chemoarchitecture, crucially determine what particular functions that area can perform (Passingham et al. 2002). Conversely, each particular cortical module is probably characterized by a unique set of input and output connections, which is supported by statistical analyses of cortical connectivity in the primate (Young 1993) and feline (Scannell et al. 1995) cortex. More generally, cortical modules and connections between those actually reflect functional segregation and functional integration, respectively (Friston 2002). In sum, a cortical module of functional specialization is likely to be defined by the intersection of regionally specific microstructure and connectivity patterns. 






SPECIAL

Functional
specialization emphasizes that microanatomically distinguishable brain
regions realize distinct functions (Brodmann 1909; Zeki 1978). Indeed,
each microanatomical cortical area has a unique connectional, and thus
perhaps functional, profile in cats (Scannell et al. 1995), monkeys
(Young et al., 1993), and humans (Passingham et al., 2002).


Consistently, early lesion studies of induced behavioral
impairments in rats challenged the dogmatic localization of brain
function (Franz \& Lashley, 1917). Analogously in humans, even
extensive prefrontal lesions do not necessarily become apparant in
everyday behavior (Mesulam, 1990). An implicit preference for
functional specialization might therefore hinder consensus in
fundamental, still unresolved questions in neuroscience, such as the
relation between cognition and emotion (cf. Pessoa 2010, Overwalle
2011).

locally characteristic neuronal populations




INTEGRAL

Hebb, D. O. The Organization Of Behaviour (Wiley,
1949).

Only recently we have shifted the main interpretational focus
from single region to macroscopical networks
in systems neuroscience \citep{yuste2015}.


eemrgent properties emerge from the interaction

In
contrast, functional integration emphasizes that function is an
emergent property of collaborating areas. In fact, the higher the
hierarchical level, the more one cortical area connects to different
large-scale networks, which makes their boundaries less clear
(Yeo/Buckner 2011). The above network-network models naturally embody
functional integration as they capture (partially overlapping)
patterns of spatial coherence.

Systems neuroscience has established existence of a
set of fluctuating yet robust brain networks in humans. 


Identical neural networks have reappeared across cognitive domains
using diverging methods. These observations prompted widely-adopted
notions, including the ``default-mode network'' \cite{raichle2001} (DMN; Raichle et al.,
2001), ``salience network'' (SN; Seeley et al., 2007), and ``dorsal
attention network'' (DAN; Corbetta and Shulmann,
2002). Developmentally, such large-scale networks emerge during late
fetal growth (Doria et al., 2010), before cognitive capacities mature
in childhood. In adults, nodes of a same cohesive network have more
similar functional profiles than nodes from different networks
(Anderson et al., 2013).
This
suggests network-network relationships as an under-appreciated unit of
functional brain organization.



Onset of a given cognitive task might induce
characteristic changes in functional coupling between large-scale
networks. Indeed, a working-memory task entailed increase in metabolic
activity in DAN regions but decrease in default-mode regions (Fransson
et al., 2006). Notably, the functional connectivity did not change
significantly within either DAN or DMN. Additionally, during auditory
event transitions in another experimental fMRI study, both DAN and SN
increased in activity, whereas the DMN decreased in activity
(Sridharan et al., 2008). 
 

Such mediation between supraordinate networks
has been proposed to involve the right anterior insula (Sridharan et
al., 2008) and right temporo-parietal junction (Bzdok et al.,
2013).
The relevance of networks
possibly extends to psychiatric, neurological, and neurodegenerative
disorders (Seeley et al., 2009; Menon 2011).







STUDY

The present investigation reframed network-network dynamics as a
multivariate statistical learning problem.
capitalized on two independent, large datasets (n=500 and n=80) with
18 typical neuroimaging tasks each, covering a spectrum of humans'
cognitive apparatus. 

determined a plausible, quantitative
combination of the brain signal's modes of variation. This allowed
avoiding the curse of dimensionality and making the machine-learning
results human-interpretable by quantitative association with
psychological concepts.

neurobiologically motivated restrictions to complexity circumvented the curse of dimensionality and allowed for useful, simplified views on brain function.



\section{Methods}
%
\paragraph{Data.}
As the currently biggest openly-accessible reference dataset,
we chose resources from the Human Connectome Project (HCP)
\cite{barch2013}.
Neuroimaging task data with labels of ongoing cognitive processes
were drawn from 500
healthy HCP participants (cf. Appendix for details on datasets).
18 HCP tasks 
were selected that are known to elicit reliable neural activity
across participants (Table \ref{table_tasks}).
In sum, the HCP task data incorporated 8650 first-level activity maps
from 18 diverse paradigms administered to 498 participants (2 removed
due to incomplete data).
All maps were resampled to a common $60\times72\times60$ space of
3mm isotropic voxels and gray-matter masked (at least 10\% tissue
probability).
The supervised analyses were thus based on labeled HCP task maps with
79,941 voxels of interest representing z-values in gray matter.

\begin{table}[h]
  \resizebox{0.98\textwidth}{!}{%
  \begin{tabular}{l|l|l}
    \hline
  {\bf Cognitive Task} & {\bf Stimuli}                         & {\bf Instruction for participants}                                                \\ \hline
  1 Reward             & \multirow{2}{*}{Card game}            & \multirow{2}{*}{Guess the number of a mystery card for gain/loss of money}        \\ \cline{1-1}
  2 Punish             &                                       &                                                                                   \\ \hline
  3 Shapes             & Shape pictures                        & Decide which of two shapes matches another shape geometrically                    \\ \hline
  4 Faces              & Face pictures                         & Decide which of two faces matches another face emotionally                        \\ \hline
  5 Random             & \multirow{2}{*}{Videos with objects}  & \multirow{2}{*}{Decide whether the objects act randomly or intentionally} \\ \cline{1-1}
  6 Theory of mind     &                                       &                                                                                   \\ \hline
  7 Mathematics        & Spoken numbers                        & Complete addition and subtraction problems                                        \\ \hline
  8 Language           & Auditory stories                      & Choose answer about the topic of the story                                        \\ \hline
  9 Tongue movement    & \multirow{3}{*}{Visual cues}          & Move tongue                                                                       \\ \cline{1-1} \cline{3-3} 
  10 Food movement     &                                       & Squeezing of the left or right toe                                                \\ \cline{1-1} \cline{3-3} 
  11 Hand movement     &                                       & Tapping of the left or right finger                                               \\ \hline
  12 Matching          & \multirow{2}{*}{Shapes with textures} & Decide whether two objects match in shape or texture                             \\ \cline{1-1} \cline{3-3} 
  13 Relations         &                                       & Decide whether object pairs differ both along either shape or texture             \\ \hline
  14 View Bodies       & Pictures                              & Passive watching                                                                   \\ \hline
  15 View Faces        & Pictures                              & Passive watching                                                                   \\ \hline
  16 View Places       & Pictures                              & Passive watching                                                                   \\ \hline
  17 View Tools        & Pictures                              & Passive watching                                                                   \\ \hline
  18 Two-Back          & Various pictures                      & Indicate whether current stimulus is the same as two items earlier                \\ \hline
  \end{tabular}
}
\vspace{-0.2cm}
\caption{\textbf{Description of psychological tasks to predict.}}
\label{table_tasks}
\end{table}

These labeled data were complemented by unlabeled activity maps
from HCP acquisitions of unconstrained resting-state activity
\cite{smith2013resting}.
These reflect brain activity in the absence of controlled thought.
In sum, the HCP rest data concatenated
8000 unlabeled, noise-cleaned rest maps with
40 brain maps from each of 200 randomly selected participants.

We were further interested in the utility of the
optimized low-rank projection
in one task dataset for dimensionality reduction in another task dataset.
To this end, the HCP-derived network decompositions were used as preliminary
step in the classification problem of another large sample.
The ARCHI dataset \cite{pinel07} provides activity maps from
diverse experimental tasks, including auditory and visual perception, motor action,
reading, language comprehension and mental calculation.
Analogous to HCP data, the second task dataset thus incorporated 1404
labeled, grey-matter masked, and z-scored activity maps
from 18 diverse tasks acquired in 78 participants.





sparse statistical models have only few nonzero parameters












\paragraph{Implementation.}
The analyses were performed in Python.
We used \textit{nilearn} to handle
the large quantities of neuroimaging data 
\cite{abrah14}
and
\textit{Theano} for automatic, numerically stable
differentiation of symbolic computation graphs
\cite{bastien2012theano, bergstra2010theano}.
All Python scripts that generated the results are
accessible online for reproducibility and reuse
% (\url{http://github.com/anonymous/anonymous}).
(\url{http://github.com/banilo/nips2015}).



\section{Experimental Results}
\paragraph{Serial versus parallel structure discovery and classification.}



\section{Discussion}

neurobiologically motivated restrictions to complexity circumvented the curse of dimensionality and allowed for useful, simplified views on brain function.






\paragraph{Acknowledgment}
{\small The research leading to these results has received funding from the
European Union Seventh Framework Programme (FP7/2007-2013)
under grant agreement no. 604102 (Human Brain Project).
Data were provided by the Human Connectome Project.
Further support was received from
the German National Academic Foundation (D.B.)
and the MetaMRI associated team (B.T., G.V.).
}

\small
\bibliographystyle{splncs03}
\bibliography{paper_refs}



\end{document}

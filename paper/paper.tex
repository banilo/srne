\documentclass{article} % For LaTeX2e
\usepackage{nips14submit_e,times}
\usepackage{hyperref}
\usepackage{url}
\usepackage{amsmath,amsfonts,amsthm}
\usepackage{bbm}
\usepackage{algorithm,algorithmic}
\usepackage{graphicx}
\usepackage{bm}
\usepackage{bbm}
\usepackage[titletoc]{appendix}
\usepackage{wrapfig}
\usepackage{afterpage}
\usepackage{amssymb}
\usepackage{booktabs}
\usepackage{ulem}
\usepackage{multirow}
\usepackage{comment}

\def\B#1{\bm{#1}}
%\def\B#1{\mathbf{#1}}
\def\trans{\mathsf{T}}
\DeclareMathOperator*{\argmin}{arg\,min}

%\renewcommand{\labelitemi}{--}

\newtheorem{theorem}{Theorem} \newtheorem{lemma}[theorem]{Lemma}
\newtheorem{proposition}[theorem]{Proposition}
\newtheorem{corollary}[theorem]{Corollary}
\newtheorem{definition}[theorem]{Definition}
\newtheorem{remark}{Remark}

%%%%%%%%%%%%%%%%%%%%%%%%%%%%%%%%%%%%%%%%%%%%%%%%%%%%%%%%%%%%%%%%%%%%%%%%%%%%%%%

\title{Region-network sparsity priors for\\
high-dimensional inference in fMRI data}

\newcommand{\fix}{\marginpar{FIX}}
\newcommand{\new}{\marginpar{NEW}}
\DeclareMathOperator{\proj}{proj}
\DeclareMathOperator{\softmax}{softmax}
\DeclareMathOperator{\prox}{prox}
\DeclareMathOperator{\Prox}{Prox}
\DeclareMathOperator{\im}{im}

% macros from michael's .tex
\DeclareMathOperator{\dist}{dist} % The distance.
\DeclareMathOperator{\argmin}{argmin}
\DeclareMathOperator{\argmax}{argmax}
\DeclareMathOperator{\Id}{Id}
\DeclareMathOperator{\abs}{abs}
\newcommand{\R}{\mathbb{R}}
\newcommand{\N}{\mathbb{N}}
\newtheorem{thm}{Theorem}[section]
\newtheorem{prop}[thm]{Proposition}
\newtheorem{lem}[thm]{Lemma}
\newtheorem{cor}[thm]{Corollary}


\newcommand{\suggestadd}[1]{{\color{blue} #1}}
\newcommand{\suggestremove}[1]{{\color{red} \sout{#1}}}

% \nipsfinalcopy % Uncomment for camera-ready version
\nipsfinaltrue
%%%%%%%%%%%%%%%%%%%%%%%%%%%%%%%%%%%%%%%%%%%%%%%%%%%%%%%%%%%%%%%%%%%%%%%%%%%%%%%

\begin{document}

\author{Danilo Bzdok, Karl Zilles, Michael Eickenberg,
  Ga\"el Varoquaux, Bertrand Thirion\\
  Department of Psychiatry, Psychotherapy and Psychosomatics, RWTH Aachen, Germany\\
  INRIA, Parietal team, Saclay, France\\
  CEA, Neurospin, Gif-sur-Yvette, France\\
  firstname.lastname@inria.fr}

\maketitle

\begin{abstract}
% Imaging neuroscience links human behavior to aspects of brain
% biology in ever-increasing datasets.


\textbf{\\keywords}:
Brain Imaging, Functional Specialization, Functional Integration,
Sparsity-inducing Norms, Structured Sparsity,
Numerical Optimization, Systems Neuroscience

\end{abstract}



\section{Introduction}
% sparsity
Many quantitative scientific domains underwent a
recent passage from the classical regime (i.e., "long data")  to
the high-dimensional regime (i.e., "wide data")
\cite{jordan2015massive}.
Also in the brain imaging domain,
many contemporary methods for acquiring brain signals yield
more variables per observation than
total observations available in a data sample.
This scenario challenges many statistical estimators from
classical statistics.
For instance,
linear regression models without additional assumptions
yield an infinity of possible coefficients
and, thus, no solution.
%
\textit{Sparsity} assumptions have consequently been
introduced and made many ill-posed estimation problems tractable
\cite{buhlmann2011statistics, hastie2015statistical}.
Domain-specific structure can be imposed on the 
statistical estimation,
thus preassuming that variables have unequal importance
and follow existing knowledge
\cite{bach2012optimization}.
Sparsified supervised and unsupervised
learning algorithms have proven to yield
statistical relationships that can be readily
estimated, reproduced, and interpreted
\cite{giraud2014introduction}.
%
Yet, what neurobiological structure suggests itself
to successfully tackle the
\textit{curse of dimensionality} in
functional neuroimaging research?



% specialization & integration
Concepts on human brain organization have long been torn
between the two extremes
\textit{functional specialization} and \textit{functional integration}.
Functional specialization emphasizes that microscopically distinguishable
brain regions probably solve distinct classes of computational processes
\cite{kanwisher2010functional}.
Functional integration, in turn, emphasizes that brain function
is probably enabled by complex connections between these
distinct brain regions \cite{sporns14nn}.
%
These notions were predominantly derived from
invasive examination of \textit{structure} (i.e., histological preparation),
\textit{connectivity}, (i.e., axonal tracing),
and \textit{functional properties}
(i.e., single cell recordings) in same animals.
Regarding functional segregation into specialized regions,
early histological investigations into the microscopic heterogeneity of
the human cerebral cortex have resulted
in several detailed anatomical maps
\cite{brodmann1909vergleichende, vogt1919allgemeine}.
Regarding axonal connections,
each such cortical area has been observed
to possess a unique set of incoming and outgoing connections
\cite{passingham2002, young93monkey, scannell95cat}.
%
Both
local cyto- and chemoarchitectonic infrastructure
and its unique global connectivity profile
together realize computational processes (i.e., function).
While
cortical modules versus connections between those
reflect 
functional specialization versus functional integration
\cite{friston2002beyond, mesulam_sensation},
both architectural principles are conceptually inextricable when
explaining the emergence of mental operations
\cite{tononi1998complexity, saygin2012}.



% SPECIAL
Functional specialization has been
explored and interpreted based on many different research methods.
%
Single cell recordings and microscopic examination
revealed, for instance, the
specialization in the visual cortex into V1, V2, V3, V3A, and V4
\cite{hubel1962receptive, zeki1978functional}.
Tissue lesion of the mid-fusiform gyrus of the visual system
were frequently reported to impair
recognition of others' identity from faces
\cite{iaria2008contrib}.
The localization of
sensory, motor, and emotional cognitive functions to cortical areas
in the living brain
has later been enabled by
non-invasive brain imaging with
functional magnetic resonance imaging (fMRI) and
positron emission tomography (PET)
\cite{fristen1997imaging}.
Further,
radioactive mapping of neurotransmitter receptor distributions
rendered accessible yet another
local characteristic of neuronal populations
\cite{zilles2009receptor}.
% allude to our many-regions layer in our methods
In the computational era,
automatic clustering methods are increasingly employed to
regionally differentiate the cerebral cortex,
which can partly be more fine-grained than
classical cytoarchitectonic borders
\cite{behrens03, cbp2015review}.
Today,
high-thoughput computing enables 
ultahigh-resolution 3D models of brain anatomy
at macroscopical to near-cellular scale
\cite{amunts2013bigbrain}.
%
As a crucial common point,
all these methodological approaches
yield neuroscientific findings
that are naturally interpreted according to
non-overlapping, discrete region compartments
as the basic architectonic framework of brain organization.



% INTEGRAL
It is more recent
that the main interpretational focus has shifted
from circumscribed regions to network stratifications
in systems neuroscience \cite{yuste2015, stephan_dys}.
The higher the processing hierarchy level,
the more a cortical area is known to
connect to different large-scale networks
\cite{yeo11}.
%
Invasive axonal tracing studies in monkeys were complemented
by in-vivo diffusion MRI tractography in humans
as a now frequently employed method to
outline fiber bundles between brain regions
\cite{jbabdi2013long}.
Besides analyses of
electrophysiological oscillations
\cite{buzsaki2004neuronal}
and
graph-theoretical properties \cite{bullmore2009complex},
studies of
functional connectivity \cite{buckner2013opportunities} and
independent component analysis (ICA) \cite{beckmann2005}
became the workhorses of network discovery
in functional neuroimaging.
These revealed the important implication of
canonical brain networks across cognitive domains,
including the so-called
``default-mode network'' \cite{raichle2001pnas},
``salience network'' \cite{seeley2007dissociable},
and ``dorsal attention network'' \cite{corbettashul2008}. 
Characteristic changes in the configuration of
these large-scale networks
were repeatedly observed to be induced
by the onset of given cognitive tasks \cite{fransson2006}.
Such task-induced mechanisms orchastrating supraordinate networks
might be subserved by the right anterior insula \cite{sridh2008}
and temporo-parietal junction \cite{bzdok2013tpj}.
%
Ultimately,
interpretation of findings from all these methods naturally embraces
cross-regional integration by
overlapping network compartments
as the basic architectonic framework of brain organization,
in stark constrast to methods examining regional specialization.



% study
Building on these interpretational traditions,
the present study proposes to incorporate
known architectonic aspects from both
anatomical segregation and integration
into unified classification and regression algorithms
by means  of structured sparsity assumptions.
The "true" relative importance of 
local region compartments and global network compartments
is typically unknown
but
probably varying in degree
across diverse neuroscientific questions.
Recent advances in multivariate statistical learning techniques
enable such adaptive estimators with
neurobiologically plausible region and network priors.
Using a large reference dataset,
we demonstrated that domain-informed supervised models
gracefully handle the curse of dimensionality,
yield many more human-interpretable results,
and generalize better to new samples
than domain-na\"ive models.

higher-order prior knowledge

structured sparsity
extend trend in stastical modelling from
genetics \cite{rapaport2008classification, kim2012tree},
natural image processing \cite{harzallah2009combining}


to brain imaging domain.


\section{Methods}
%
\paragraph{Rationale}

we need to inject domain knowledge into
statistical estimations to harness the curse of dimensionality.
two neurobiological design principles

integrative processes


should be able to estimate voxel level
while taking into account known supravoxel structure.
is instrumental in
Developmentally, such large-scale networks emerge during late
fetal growth (Doria et al., 2010), before cognitive capacities mature
in childhood. 

In adults, nodes of a same cohesive network have more
similar functional profiles than nodes from different networks
(Anderson et al., 2013).

data exhibit natural correlations between neighboring voxels forming clusters

neurobiologically motivated restrictions to complexity circumvented
the curse of dimensionality
three-dimensional spatial arrangement that respects
the functional anatomy of the brain
not ignore the spatial configuration

incorporate rich prior knowledge

If meaningful structures exist,
ÇÇçwe show that one can take advantage of such structures


\paragraph{Problem formulation}

Sparse linear models
encode geometric prior information
topology
local sets of voxels

structured regularization
We might therefore gain in the quality of the factors
induced by enforcing directly this a priori 



l1 = unstructured sparsity-inducing penalty

for every column u of U, it compute a column v of V solving

we aim at learning a weight vector w ∈ Rp and an intercept b ∈ R
such that the prediction of y can be based on the value of w⊤x + b.

We omit a bias term, since the data were mean-centered
and unit-variance scaled.
The scalar b is not particularly informative

however the vector w corresponds to a volume that
can be represented in brain space as a volume

hierarchial tree = more generally into a directed acyclic graph

more precisely, we denote by X ∈ Rn×p the design matrix
assembled from n fMRI volumes and by y ∈ Rn the corresponding n targets.
In other words, each row of X is a p-dimensional sample,
i.e., an activation map of p voxels related to one stimulus presentation.
for visualization of the predictive pattern of voxels. 

Learning the parameters (w, b) remains challenging
since the number of features (104 to 105 voxels) exceeds
by far the number of samples (a few hundreds of volumes). 

The scalar b is not particularly informative,
however the vector w corresponds to a volume that
can be represented in brain space as a volume
for visualization of the predictive pattern of voxels.

each row of X is a p-dimensional sample,
i.e., an activation map of p voxels related to one stimulus presentation.

To address this issue, dimensionality reduction attempts to
find a low dimensional subspace that concentrates
as much of the predictive power of
the original set as possible for the problem at hand.
-> we do not want to do preliminary feature selection or
dimensionality reduction
or feature agglomeration because we want to fit one model parameter
to each brain voxel for maximal interpretability
This corresponds to discarding some columns of X.

The essential shortcoming of the Elastic net is that
it does not take into account the spatial structure of the data,
which is crucial in this context

Craddock clusters are often used for feature agglomeration
into parcels
-> exploits only a part of the data

dual-level spatial structure
sparse hierarchical regularization
structured sparsity-inducing regularization
the root of the tree T is the unique cluster that gathers all the voxels,

It is a generalization of the traditional $\ell_1$-norm
$\Omega(\mathbf{w}) = \sum_{j=1}^p|\mathbf{w}_j|$
ignores structure


\cite{jenatton2011multi}

\textit{structured sparsity}

\cite{huang2011learning, morales2010family, jenatton2011structured}


a node $j$ of $\mathcal{T}$,
we denote by $g_j \subseteq \{1,..,q\}$ the set of indices that record
all the descendants of $j$ in $\mathcal{T}$

the family of sparsity-inducing norms has recently been extended
by hierarchical sparsity penalty terms
\cite{zhao2009composite}.


\begin{equation}
  \Omega(\mathbf{w}) = \sum_{g \in G}||\mathbf{w}_g||_2 = \sum_{g \in G}\sqrt{\sum_{j \in g}\mathbf{w}_j^2}
\end{equation}

For example, when G is the set of all singletons, Ω is the
usual l1 norm (assuming that all the weights are equal to 1).

l1/l2 mixed norm is convex


Discarding coefficients belonging to a network group will naturally enforce
discarding the coefficients belonging to each of its descendent region groups.
Conversely,
variable selection of a network group will also enforce
selection of all voxel of its descendent group regions.
Single region groups can however be set to zero (unselected)
or non-zero (selected)
without analogous effect on the parent network group.




At the between-group level,...
$\Omega$ exerts $\ell_1$-like variable selection on
the (||\mathbf{w}_g||_2)_{g\inG} groups,
yielding a maximum of $g \in G$ to be zeroed out
\cite{jenatton2011structured}.
The important consequence is that also all descentes of such a zeroed
group $g \in G$ will be descarded.
Conversely,
if one group $g$ is selected,
then all the ancestral groups will also be selected.
Thus, statistical estimation will be improved by enticing
entire voxel sets to be selected or discarded as predictive, 
although one individual coefficient is computed for each voxel.

-> it is a $(\ell_1, \ell_2)$-mixed norm
-> between-group sparsity effect by l1
-> within-group shrinkage effect by l2


\begin{equation}
  \Omega(\mathbf{w}) = \sum_{g \in G} \eta_g ||\mathbf{w}_g||_2
\end{equation}

(\eta_g)_{g \in G} are positive weights for the groups


fit to the data is measured through
a convex loss function (w, b) 􏰀→ L(y, X, w, b) ∈ R+. 


\paragraph{Classification}

logistic loss function

\begin{equation}
  P(y=k|x, W, b) = \frac{exp\{x^Tw^k + b_k\}}{\sum_{m=1}^cexp\{x^Tw^m + b_m\}}

  \argmin \frac{1}{2}||u-v||_2^2 + \lambda\Omega(\mathbf{w})

\end{equation}

\lambda > 0.

bias is omitted because \mathbf{X} and \mathbf{y}
are mean-centered and unit-variance scaled.

In this setting, and given a new fMRI volume x,
we make predictions by choosing the label that maximizes
the class-conditional probabil- ities (3.1), that is, argmaxk∈{1,...,c}Prob(y = k|x; W∗, b∗)

One-versus-rest scheme


\paragraph{Regression}

squared error as loss

\begin{equation}
  \argmin \frac{1}{2}||\mathbf{y - Xw}||_2^2 + \lambda\Omega(\mathbf{w})
\end{equation}

\lambda > 0.

Prediction for a new fMRI volume x is then simply performed by computing
the dot product x⊤w∗

\paragraph{Numerical optimization}

Difficult because high-dimensional setting

empirical risk minimization was performed by


The intercept $b$ is left unregularized




\begin{comment}


\paragraph{Implementation.}
The analyses were performed in Python.
We used \textit{nilearn} to handle
the large quantities of neuroimaging data 
\cite{abrah14}
and
\textit{Theano} for automatic, numerically stable
differentiation of symbolic computation graphs
\cite{bastien2012theano, bergstra2010theano}.
All Python scripts that generated the results are
accessible online for reproducibility and reuse
% (\url{http://github.com/anonymous/anonymous}).
(\url{http://github.com/banilo/nips2015}).
  

%
\paragraph{Data.}
As the currently biggest openly-accessible reference dataset,
we chose resources from the Human Connectome Project (HCP)
\cite{barch2013}.
Neuroimaging task data with labels of ongoing cognitive processes
were drawn from 500
healthy HCP participants (cf. Appendix for details on datasets).
18 HCP tasks 
were selected that are known to elicit reliable neural activity
across participants (Table \ref{table_tasks}).
In sum, the HCP task data incorporated 8650 first-level activity maps
from 18 diverse paradigms administered to 498 participants (2 removed
due to incomplete data).
All maps were resampled to a common $60\times72\times60$ space of
3mm isotropic voxels and gray-matter masked (at least 10\% tissue
probability).
The supervised analyses were thus based on labeled HCP task maps with
79,941 voxels of interest representing z-values in gray matter.

\begin{table}[h]
  \resizebox{0.98\textwidth}{!}{%
  \begin{tabular}{l|l|l}
    \hline
  {\bf Cognitive Task} & {\bf Stimuli}                         & {\bf Instruction for participants}                                                \\ \hline
  1 Reward             & \multirow{2}{*}{Card game}            & \multirow{2}{*}{Guess the number of a mystery card for gain/loss of money}        \\ \cline{1-1}
  2 Punish             &                                       &                                                                                   \\ \hline
  3 Shapes             & Shape pictures                        & Decide which of two shapes matches another shape geometrically                    \\ \hline
  4 Faces              & Face pictures                         & Decide which of two faces matches another face emotionally                        \\ \hline
  5 Random             & \multirow{2}{*}{Videos with objects}  & \multirow{2}{*}{Decide whether the objects act randomly or intentionally} \\ \cline{1-1}
  6 Theory of mind     &                                       &                                                                                   \\ \hline
  7 Mathematics        & Spoken numbers                        & Complete addition and subtraction problems                                        \\ \hline
  8 Language           & Auditory stories                      & Choose answer about the topic of the story                                        \\ \hline
  9 Tongue movement    & \multirow{3}{*}{Visual cues}          & Move tongue                                                                       \\ \cline{1-1} \cline{3-3} 
  10 Food movement     &                                       & Squeezing of the left or right toe                                                \\ \cline{1-1} \cline{3-3} 
  11 Hand movement     &                                       & Tapping of the left or right finger                                               \\ \hline
  12 Matching          & \multirow{2}{*}{Shapes with textures} & Decide whether two objects match in shape or texture                             \\ \cline{1-1} \cline{3-3} 
  13 Relations         &                                       & Decide whether object pairs differ both along either shape or texture             \\ \hline
  14 View Bodies       & Pictures                              & Passive watching                                                                   \\ \hline
  15 View Faces        & Pictures                              & Passive watching                                                                   \\ \hline
  16 View Places       & Pictures                              & Passive watching                                                                   \\ \hline
  17 View Tools        & Pictures                              & Passive watching                                                                   \\ \hline
  18 Two-Back          & Various pictures                      & Indicate whether current stimulus is the same as two items earlier                \\ \hline
  \end{tabular}
}
\vspace{-0.2cm}
\caption{\textbf{Description of psychological tasks to predict.}}
\label{table_tasks}
\end{table}

These labeled data were complemented by unlabeled activity maps
from HCP acquisitions of unconstrained resting-state activity
\cite{smith2013resting}.
These reflect brain activity in the absence of controlled thought.
In sum, the HCP rest data concatenated
8000 unlabeled, noise-cleaned rest maps with
40 brain maps from each of 200 randomly selected participants.

We were further interested in the utility of the
optimized low-rank projection
in one task dataset for dimensionality reduction in another task dataset.
To this end, the HCP-derived network decompositions were used as preliminary
step in the classification problem of another large sample.
The ARCHI dataset \cite{pinel07} provides activity maps from
diverse experimental tasks, including auditory and visual perception, motor action,
reading, language comprehension and mental calculation.
Analogous to HCP data, the second task dataset thus incorporated 1404
labeled, grey-matter masked, and z-scored activity maps
from 18 diverse tasks acquired in 78 participants.





sparse statistical models have only few nonzero parameters








\section{Experimental Results}
\paragraph{Serial versus parallel structure discovery and classification.}



\section{Discussion}

exploit known structure for variable selection and prediction
in classification and regression


Functional specialization and functional integration
in the human brain are
probably both mechanistically relevant for emerging behavior.
However, there is a scarcity of statistical methods
that concomitantly relate brain regions and networks to
an organism's repertoire of mental operations.


We assume regions and networks but can estimate 
importance of each voxel.

neurobiologically motivated restrictions to
complexity circumvented the curse of dimensionality and allowed
for useful, simplified views on brain function.


The same classification performance can be reached by
many different models -> we naturally want to priviledge those
among the equally performing models that are closest to
what we know about the brain

-> interpreta bility and accuracy typically exclude each other
(cf. Hastie)

alternatives would be dimensionality reduction
and dictionary learning


curse of dimensionality has classically been circumvented
by treating a high-dimensional problem
as a mass-univariate problem.
Problem was postponed to the inference step
this is why we never solved the multiple comparisons problem in fMRI

regularization with sparsity penalties
restrictions to complexity according to neurobiological facts
-> bias-variance tradeoff: negigible increase in bias but large decrease in variance
-> VC: decrease complexitiy capactiy -> better chance for generalization out of sample

multiple comparisons problem has the curse of dimensionality
as its close correspondence in statistical learning
sparsity -> solves the curse at the estimation level



OUTLOOK:
robust high-dimensional inference will be key
to prediction in single individual,
such as for personalized medicine
\cite{gabrieli2015prediction}

More generally, our regularization scheme could also be used for various
learning tasks, as soon
as prior knowledge on the structure of the sparse representation
is available, e.g.,
for multiple kernel learning (Micchelli and Pontil, 2006),
multi-task learning (Argyriou et al., 2008; Obozinski et al., 2009;
Kim and Xing, 2009) and sparse matrix factorization
problems (Mairal et al., 2010; Jenatton et al., 2010).


this type of prior knowledge is of clear interest for the spatially and tempo-
rally structured data typical in bioinformatics, computer vision and neuroscience applications

structured sparsity can even be used for
unsupervised data modelling,
such as in PCA \cite{jenatton2009structured}



\paragraph{Acknowledgment}
{\small The research leading to these results has received funding from the
European Union Seventh Framework Programme (FP7/2007-2013)
under grant agreement no. 604102 (Human Brain Project).
Data were provided by the Human Connectome Project.
Further support was received from
the German National Academic Foundation (D.B.)
and the MetaMRI associated team (B.T., G.V.).
}


\end{comment}
  
\small
% \bibliographystyle{plainnat}
\bibliographystyle{splncs03}
\bibliography{paper_refs}



\end{document}
